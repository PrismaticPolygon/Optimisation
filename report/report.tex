% !TEX TS-program = pdflatex
% !TEX encoding = UTF-8 Unicode

% This is a simple template for a LaTeX document using the "article" class.
% See "book", "report", "letter" for other types of document.

\documentclass[11pt]{article} % use larger type; default would be 10pt

\usepackage[utf8]{inputenc} % set input encoding (not needed with XeLaTeX)

%%% Examples of Article customizations
% These packages are optional, depending whether you want the features they provide.
% See the LaTeX Companion or other references for full information.

%%% PAGE DIMENSIONS
\usepackage{geometry} % to change the page dimensions
\geometry{a4paper} % or letterpaper (US) or a5paper or....
% \geometry{margin=2in} % for example, change the margins to 2 inches all round
% \geometry{landscape} % set up the page for landscape
%   read geometry.pdf for detailed page layout information

\usepackage{graphicx} % support the \includegraphics command and options

% \usepackage[parfill]{parskip} % Activate to begin paragraphs with an empty line rather than an indent

%%% PACKAGES
\usepackage{booktabs} % for much better looking tables
\usepackage{array} % for better arrays (eg matrices) in maths
\usepackage{paralist} % very flexible & customisable lists (eg. enumerate/itemize, etc.)
\usepackage{verbatim} % adds environment for commenting out blocks of text & for better verbatim
\usepackage{subfig} % make it possible to include more than one captioned figure/table in a single float
% These packages are all incorporated in the memoir class to one degree or another...

\usepackage{amsmath}
\usepackage{algorithm}
\usepackage{algpseudocode}

\usepackage{xcolor,colortbl}
\definecolor{green}{rgb}{0.1,0.1,0.1}
\newcommand{\done}{\cellcolor{teal}-3}  %{0.9}
\newcommand{\hcyan}[1]{{\color{teal} #1}}

%%% HEADERS & FOOTERS
\usepackage{fancyhdr} % This should be set AFTER setting up the page geometry
\pagestyle{fancy} % options: empty , plain , fancy
\renewcommand{\headrulewidth}{0pt} % customise the layout...
\lhead{}\chead{}\rhead{}
\lfoot{}\cfoot{\thepage}\rfoot{}

%%% SECTION TITLE APPEARANCE
\usepackage{sectsty}
\allsectionsfont{\sffamily\mdseries\upshape} % (See the fntguide.pdf for font help)
% (This matches ConTeXt defaults)

%%% ToC (table of contents) APPEARANCE
\usepackage[nottoc,notlof,notlot]{tocbibind} % Put the bibliography in the ToC
\usepackage[titles,subfigure]{tocloft} % Alter the style of the Table of Contents
\renewcommand{\cftsecfont}{\rmfamily\mdseries\upshape}
\renewcommand{\cftsecpagefont}{\rmfamily\mdseries\upshape} % No bold!


%%% END Article customizations

\title{Optimisation}
\author{ffgt86}
%\date{} % Activate to display a given date or no date (if empty),
         % otherwise the current date is printed 

\begin{document}
\maketitle

\section*{Question 1}

The first step is to convert $min(x_1 - 3x_2)$ to a maximisation problem. This is trivial:

\begin{center}

$min(x_1 - 3x_2) = max(-x_1 + 3x_2)$

\end{center}

and can be rewritten as:

\begin{center}

$x_1 - 3x_2 + P = 0$

\end{center}

Constraints must also be converted to equalities, with the addition of the slack variables $s_1$, $s_2$, and $s_3$:

\begin{center}

$x_1 - x_2 \leq 1 \Rightarrow x_1 - x_2 + s_1 = 1$ \\
$x_1 - x_2 \geq -1 \Rightarrow -x_1 + x_2 + s_2 = 1$ \\
$2x_1 - x_2 \leq 3 \Rightarrow 2x_1 - x_2 + s_3 = 3$ 

\end{center}

The first tableau can now be constructed:

\begin{center}

\begin{tabular}{ c | c c c c c | c | c }
    & $x_1$ & $x_2$ & $s_1$ & $s_2$ & $s_3$ & b & t \\ \hline
  1 & 1 & -3 & 0 & 0 & 0 & 0 & \\  \hline
  0 & 1 & -1 & 1 & 0 & 0 & 1 & -1\\
  0 & -1 & 1 & 0 & 1 & 0 & 1 & 1\\
  0 & 2 & -1 & 0 & 0 & 1 & 3 & -3\\
\end{tabular}

\end{center}

As column $x_2$ contains the smallest negative number in the first row ($-3$), we calculate pivot values, in column $t$, using $x_2$. As $1$ is the smallest non-negative value, we pivot around the third element of column $x_2$, $1$. Using the following row operations:

\begin{center}

$R_1 = R_1 + 3R_3$ \\
$R_2 = R_2 + R_3$ \\
$R_4 = R_4 + R_3$ \\

\end{center}

the next tableau can be constructed:

\begin{center}

\begin{tabular}{ c | c c c c c | c | c }
    & $x_1$ & $x_2$ & $s_1$ & $s_2$ & $s_3$ & b & t \\ \hline
  1 & -2 & 0 & 0 & 3 & 0 & 3 & \\  \hline
  0 & 0 & 0 & 1 & 1 & 0 & 2 & -\\
  0 & -1 & 1 & 0 & 1 & 0 & 1 &-1\\
  0 & 1 & 0 & 0 & 1 & 1 & 4 & 4\\
\end{tabular}

\end{center}

The next pivot is the fourth element of column $x_1$, $1$, as $4$ is the smallest non-negative pivot value in $t$.  Using the following row operations:

\begin{center}

$R_1 = R_1 + 2R_4$ \\
$R_3 = R_3 + R_4$ \\

\end{center}

the final tableau can be constructed.

\begin{center}

\begin{tabular}{ c | c c c c c | c  }
    & $x_1$ & $x_2$ & $s_1$ & $s_2$ & $s_3$ & b\\ \hline
  1 & 0 & 0 & 0 & 5 & 2 & 11 \\  \hline
  0 & 0 & 0 & 1 & 1 & 0 & 2\\
  0 & 0 & 1 & 0 & 2 & 1 & 5\\
  0 & 1 & 0 & 0 & 1 & 1 & 4\\
\end{tabular}

\end{center}

As there are no remaining negative values in the top row, an optimal result has been calculated. Non-unit columns are non-basic and are therefore $0$. Reading from the tableau, the following values can be obtained:

\begin{center}

	$x_1 = 4$ \\
	$x_2 = 5$

\end{center}

Plugging these back into the original minimisation problem thus gives an optimal value:

\begin{center}

	$x_1 - 3x_2 = 4 - 3(5) = -11$

\end{center}

\clearpage

\section*{Question 2}

As the LP is provided in a canonical form, we can construct a tableau immediately in the form:

\begin{center}

\begin{tabular}{ c | c | c  }
	$1$ & $-c^T$ & $z$ \\ \hline
	$0$ & $A$ & $b$ \\
\end{tabular}

\end{center}

as follows:

\begin{center}

\begin{tabular}{ c | c c c c c c | c | c  }
    & $x_1$ & $x_2$ & $x_3$ & $x_4$ & $x_5$ & $x_6$ & $b$ & $t$\\ \hline
  1 & -2 & -1 & 1 & 0 & 0 & 0 & 0 &  \\  \hline
  0 & -2 & 1 & 1 & 1 & 0 & 0 & 1 & -0.5 \\
  0 & 1 & -1 & 0 & 0 & 1 & 0 & 2 & 2\\
  0 & 2 & -3 & -1 & 0 & 0 & 1 & 6 & 3\\
\end{tabular}

\end{center}

As column $x_1$ contains the smallest negative number in the first row ($-2$), we calculate pivot values, in column $t$, using $x_1$. As $2$ is the smallest non-negative value, we pivot around the third element of column $x_1$, $1$. Using the following row operations:

\begin{center}

$R_1 = R_1 + 2R_3$ \\
$R_2 = R_2 + 2R_1$ \\
$R_4 = R_4 - 2R_3$ \\

\end{center}

the next tableau can be constructed:

\begin{center}

\begin{tabular}{ c | c c c c c c | c | c  }
    & $x_1$ & $x_2$ & $x_3$ & $x_4$ & $x_5$ & $x_6$ & $b$ & $t$\\ \hline
  1 & 0 & -3 & 1 & 0 & 2 & 0 & 4 &  \\  \hline
  0 & 0 & -1 & 1 & 1 & 2 & 0 & 5 & -5 \\
  0 & 1 & -1 & 0 & 0 & 1 & 0 & 2 & -2\\
  0 & 0 & -1 & -1 & 0 & -2 & 1 & 2 & -2\\
\end{tabular}

\end{center}

As there are no pivot "exiting" variables, this LP is \textit{unbounded}.

\clearpage

\section*{Question 3}

For this problem, $9$ variables are necessary. Each relates to the amount of paint of one colour (cyan $C$, magenta $M$, or yellow $Y$) used in the creation of paint of another colour (red $R$, green $G$, blue $B$, or black $K$), e.g. $C_K$ denotes the number of gallons of cyan ($C$) paint used in the production of black ($K$) paint. The objective function is therefore:

\begin{center}

$max \bigl( \frac{10 \cdot (Y_R + M_R)}{2} + \frac{15 \cdot (Y_G + C_G)}{2} + \frac{25 \cdot (M_B + C_B)}{2} + \frac{25 \cdot (C_K + M_K + Y_K)}{3}\bigr)$

\end{center}

Each term is the amount of paint of a certain colour produced multiplied by the value of a gallon of paint of that colour: $10$ for $R$, $15$ for $G$, $25$ for $B$, and $25$ for $K$. The following constraints capture the limited quantities of paint available: $11$ gallons of $Y$, $10$ of $C$, and $5$ of $M$:

\begin {center}

$Y_R + Y_G + Y_K <= 11$ \\
$C_G + C_B + C_K <= 10$ \\
$M_B + M_K +M_R <= 5$ \\

\end{center}

Also necessary are constraints maintaining the correct ratios of paints used:

\begin{center}

$Y_R == M_R$ \\
$Y_G == C_G$ \\
$M_B == C_B$ \\
$C_K == M_K == Y_K$ \\

\end{center}

and constraints ensuring that the input paint volume equals the output paint volume:

\begin{center}

$Y_R + M_R ==  R$ \\
$Y_G + C_G == G$ \\
$M_B + C_B$ == B \\
$C_K + M_K + Y_K == K$ \\

\end{center}

An optimal solution produces $10$ gallons of $G$ paint and $15$ gallons of $K$ paint, using $10$ gallons of $C$ paint, $5$ gallons of $M$ paint, and $10$ gallons of $Y$ paint, for a total value of \textit{£}$525$.

\clearpage

\section*{Question 4}

\subsection*{Part A} 

This is a simple knapsack problem. Each variable $A, B, C, D, E, F$ is binary: whether or not the item was taken. This leads to the following function:

\begin{center}

$max (60A + 70B + 40C + 70D + 16E + 100F)$

\end{center}

in which the constants are the values (\textit{£}) of each item. The only constraint is equally simple: that the weight of the taken items does not exceed $20$kg:

\begin{center}

$6A + 7B + 4C + 9D + 3E + 8F <= 20$

\end{center}

in which the constants are the weights (\textit{kg}) of each item. An optimal solution is to take items $B$, $C$, and $F$, resulting in a total weight of $19$\textit{kg} and a total value of \textit{£}$210$.

\subsection*{Part B}

This part adds a new constraint:  that taking $C$ only makes sense if $D$ is also taken, but not vice versa. This can be elegantly expressed as:

\begin{center}

$D - C >= 0$

\end{center}

This condition is only unsatisfied if $D = 0$ and $C = 1$. With this constraint, an optimal solution is to take items $D$, $E$, and $F$, resulting in a total weight of $20$\textit{kg} and a total value of \textit{£}$186$.

\subsection*{Part C}

This part adds a further modification. It is now possible to exceed the $20$\textit{kg} limit, but with a penalty of \textit{£}$15$ for each \textit{kg} over. A new variable, $w$, is necessary. The objective function is modified to:

\begin{center}

$max (60A + 70B + 40C + 70D + 16E + 100F - 15w)$

\end{center}

to capture the cost of exceeding the weight limit. An additional constraint is also required:

\begin{center}

$w == 6A + 7B + 4C + 9D + 3E + 8F - 20$

\end{center}

to set $w$ to number of \textit{kg} over the weight limit the solution is.
An optimal solution is to take items $A$, $B$, and $F$, resulting in a total weight of $21$\textit{kg} and a total value of \textit{£}$215$.

\end{document}
